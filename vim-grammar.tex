\documentclass{beamer}

    \usepackage[utf8]{inputenc}
    \usepackage[T1]{fontenc}
    \usepackage[french]{babel}
    \usepackage{url}

		\usepackage{graphicx}

    \newcommand\dev{\textbf{\footnotesize{[DEV]}}}

    %\usetheme{Singapore}
		\usetheme[progressbar=frametitle]{m}

    \title[Vim Grammar]{Vim Grammar}
    \author{Rompez la solitude de votre éditeur : parlez lui !}
    \institute{Mathieu (matael) Gaborit --- HAUM}
    \date{2015}

\begin{document}

\begin{frame}
\titlepage
\end{frame}

\begin{frame}
\frametitle{Au Menu}
\tableofcontents
\end{frame}

\section{Vi-quoi ?}
\subsection{}
\frame{\tableofcontents[currentsection]}

\begin{frame}
\frametitle{Vi-quoi ?}
\begin{center}
vim
\end{center}
\pause{}

{\bf  Caractéristiques} 
\begin{itemize}
    \item prononcer {\it vi-aïe-me}
    \item éditeur de texte modal
    \item hautement personalisable/scriptable
    \item prévu pour la console
\end{itemize}

\pause{}
{\bf Signe particulier} 
\pause
\begin{center}
	\alert{Fait peur aux débutants}
\end{center}
\end{frame}

\begin{frame}
\frametitle{Pourquoi fait il peur ?}
{\bf On lui reproche}
\pause{}
\begin{itemize}
    \item raccourcis obscurs
    \item manque d'intuitivité du mode "Normal" (commande)
    \item sa vie en console
\end{itemize}

\pause{}

{\bf Pourtant...}
\pause{}
\begin{itemize}
    \item peu gourmand en ressources
    \item possibilité de le lancer en mode {\it client-serveur}
    \item capacité d'extension énorme
\end{itemize}

\pause{}
{\bf Solution pour l'apprentissage}
\pause{}
\begin{center}
Faire de la grammaire !\\
\scriptsize{...Et désactiver les flèches.}
\end{center}

\end{frame}

\section{Grammaire}
\frame{\tableofcontents[currentsection]}
\subsection{Anatomie d'une commande}
\frame{\tableofcontents[currentsubsection]}


\begin{frame}
\frametitle{Anatomie d'une commande}
\begin{center}
    \Large{<répétition><verbe><modificateur><nom>}

    \Large{<répétition><verbe><mouvement>}
\end{center}
\pause{}
\begin{description}[<+->]
    \item[nombre] Combien de fois faut il répeter la commande ?
    \item[verbe] Que veut on faire ?
    \item[modificateur] Modifie le comportement du nom
    \item[nom] Sur quoi veut on agir ?
    \item[mouvement] Jusqu'où agir ?
\end{description}
\end{frame}

\subsection{Verbes, Mouvements, Noms, Modificateurs}
\frame{\tableofcontents[currentsubsection]}

\begin{frame}
\frametitle{Verbes}

Quelle action cherche-t-on à réaliser ?
\pause{}
\begin{description}[<+->]
    \item[c] Changer
    \item[y] Copier ({\it yank})
    \item[v] Sélectioner (passage en mode {\bf v}isuel)
    \item[d] Supprimer ({\it delete})
		\item[g] Action globale
\end{description}
    
\end{frame}

\begin{frame}
\frametitle{Mouvements}

Jusqu'où veut on agir ?
\pause{}
\begin{description}[<+->]
    \item[h/j/k/l] (gauche/bas/haut/droite, à combiner avec un chiffre)
		\item[b/w] un mot en arrière/en avant ({\it beginning/word})
    \item[\$] fin de la ligne (cf {\it regexps})
    \item[\^{}] début de la ligne (cf {\it regexps })
    \item[gg/G] début/fin de fichier 
    \item[/regexp] jusqu'a matcher la {\it regexp} (vers l'avant)
    \item[?regexp] jusqu'a matcher la {\it regexp} (vers l'arrière)
\end{description}
    
\end{frame}

\begin{frame}
\frametitle{Noms}

Sur quoi veut on agir ?
\pause{}
\begin{description}[<+->]
    \item[w] un mot ({\it word})
    \item[p] un paragraphe
    \item[b] un bloc/un groupe de parenthèses
		\item[t] une balise (XML/HTML seulement, {\it tag})
\end{description}
    
\end{frame}

\begin{frame}
\frametitle{Modificateurs}

Modification des noms...
\pause{}
\begin{description}[<+->]
    \item[i] {\it inside} (travaille {\it dans} l'objet)
    \item[a] travaille autour de l'objet ({\it around}, prend une espace en plus)
    \item[t] travaille jusqu'a l'objet ({\it till})
    \item[f] comme {\it t} mais en incluant l'objet
\end{description}

\end{frame}

\subsection{Exemples}
\frame{\tableofcontents[currentsubsection]}

\begin{frame}{Exemples}
\begin{center}
\alert{2dw ou d2w}

\pause{}
Supprimer (\alert{d}) \alert{2} mots (\alert{w})

\pause{}

\alert{va"}

\pause{}
Sélectionner visuellement (\alert{v}) autour (\alert{a}) des guillemets (\alert{"})

\pause{}

\alert{ci\{}

\pause{}
Changer (\alert{c}) à l'intérieur (\alert{i}) des accolades (\alert{\{})

\end{center}
\end{frame}

\begin{frame}{Exemples}
\begin{center}

\alert{da]}

\pause{}
Supprimer (\alert{d}) autour (\alert{a}) des crochets (\alert{]})

\pause{}

\alert{c\$}

\pause{}
Changer (\alert{c}) jusqu'en fin de ligne (\alert{\$})

\pause{}

\alert{v4k}

\pause{}
Sélectionner visuellement (\alert{v}) \alert{4} lignes vers le haut (\alert{k})

\pause{}

\alert{vap:s/a/b/g (plus dur hein ?!)}

\pause{}
Sélectionner visuellement (\alert{v}) autour (\alert{a}) du paragraphe (\alert{p}) puis (\alert{:} = mode commande) remplacer tous les {\it a} par des {\it b} (\alert{s/a/b/g})
\end{center}
\end{frame}

\section{Encore !}
\subsection{}
\frame{\tableofcontents[currentsection]}

\begin{frame}
\frametitle{Etendre la grammaire}

Vim est \alert{scriptable} et il y a \alert{plein de plugins} :

\begin{center}
Possibilité de rajouter des verbes/noms à la grammaire
\end{center}
\end{frame}

\begin{frame}
\frametitle{Exemple de nouveau verbe : {\it surround}}

Entoure avec quelque chose (\alert{S}).

Disponible à : \url{https://github.com/tpope/vim-surround}

\pause{}
\begin{center}
    vipS(
    \pause{}

    (Sélectionner le paragraphe et l'entourer de parenthèses)
\end{center}
\end{frame}


\begin{frame}
\frametitle{Exemple de nouveau verbe : {\it go Comment}}

Commenter quelque chose (\alert{gc})

Disponible à : \url{https://github.com/tomtom/tcomment_vim}
\pause{}

\begin{center}
		\alert{gc/\}}
    \pause{}

    (commenter jusqu'à la prochaine accolade fermante)
\end{center}
    
\end{frame}

\section{Remerciements}
\subsection{}
\frame{\tableofcontents[currentsection]}

\begin{frame}
\frametitle{Remerciements}

{\bf Merci à }

\begin{itemize}
    \item Tous les bloggers ayant écrit là dessus et particulièrement :
    \begin{itemize}
				\item \alert{Yan Pritzker} (\url{yanpritzker.com})
				\item \alert{Jamie Curle} (\url{jamiecurle.co.uk})
        \item \alert{Jared Carroll} (\url{blog.carbonfive.com})
    \end{itemize}
		\item \alert{Bram Moolenaar} pour avoir écrit vim et \alert{Bill Joy} pour vi
		\item \alert{Vincent Jousse} pour son livre \alert{Vim pour les Humains} (\url{vimebook.com})
\end{itemize}

{\bf Licence}

\begin{center}
    Distribué sous licence CC-By-SA
    \url{http://creativecommons.org/licenses/by-sa/3.0/}

		\includegraphics[height=.5cm]{cc-by-sa.png}
\end{center}
\end{frame}

\end{document}
